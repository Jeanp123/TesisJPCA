
\subsection{Inteligencia Artificial y Procesamiento de Lenguaje Natural}

\textbf{Inteligencia Artificial (IA)}: La IA se refiere a la simulación de procesos de inteligencia humana por parte de sistemas informáticos. En el contexto de un chatbot médico, la IA se utiliza para analizar las consultas de los usuarios y proporcionar respuestas precisas y relevantes.

\textbf{Procesamiento de Lenguaje Natural (NLP)}: El NLP es una rama de la IA que se centra en la interacción entre computadoras y humanos mediante el lenguaje natural. Los modelos de NLP permiten a los chatbots entender y responder a las consultas en lenguaje humano, facilitando la comunicación natural y efectiva.

\subsection{Modelos de Machine Learning}

\textbf{Machine Learning (ML)}: Es una subdisciplina de la IA que permite a los sistemas aprender y mejorar a partir de la experiencia sin ser explícitamente programados para cada tarea. Los algoritmos de ML son fundamentales para el funcionamiento de los chatbots, ya que permiten mejorar la precisión y relevancia de las respuestas basadas en datos históricos.

\textbf{Redes Neuronales y Deep Learning}: Las redes neuronales, especialmente las redes neuronales profundas (deep learning), han revolucionado el campo del NLP. Modelos como Transformers y BERT (Bidirectional Encoder Representations from Transformers) son capaces de comprender el contexto y matices del lenguaje humano con gran precisión.

\subsection{Telemedicina y Salud Digital}

\textbf{Telemedicina}: La telemedicina se refiere al uso de tecnologías de la información para proporcionar servicios médicos a distancia. Los chatbots médicos son una extensión de la telemedicina, proporcionando asesoramiento y orientación médica sin la necesidad de una interacción cara a cara.

\textbf{Salud Digital}: La salud digital engloba todas las aplicaciones de las tecnologías digitales en el ámbito sanitario. Esto incluye desde aplicaciones móviles hasta sistemas complejos de gestión de datos de pacientes. Los chatbots médicos se sitúan en la intersección de la salud digital y la telemedicina, facilitando el acceso a información médica de calidad.