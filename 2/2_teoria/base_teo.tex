\subsection{Inteligencia Artificial (IA)}: La IA hace referencia a la imitación de los procesos de inteligencia humana por parte de sistemas de computación. En el contexto de un chatbot médico, la IA se utiliza para analizar las consultas de los usuarios y proporcionar respuestas precisas.

\subsection{Procesamiento de Lenguaje Natural (NLP)}: El NLP es una rama de la IA que se centra en la interacción entre computadoras y humanos mediante el lenguaje natural. Los modelos de NLP permiten a los chatbots entender y responder a las consultas en lenguaje humano, facilitando la comunicación natural y efectiva.

\subsection{BERT (Bidirectional Encoder Representations from Transformers):} BERT es un modelo de lenguaje preentrenado desarrollado por Google, diseñado para comprender el contexto de las palabras en ambas direcciones en una oración. Utiliza una arquitectura de transformadores bidireccional que permite a BERT capturar matices más profundos del lenguaje, mejorando significativamente el rendimiento en tareas de procesamiento del lenguaje natural, como la respuesta a preguntas y el análisis de sentimientos.

\subsection{LSTM (Long Short-Term Memory):} Las LSTM son un tipo de red neuronal recurrente diseñada para manejar problemas de aprendizaje de secuencias largas y dependencias a largo plazo. A diferencia de las redes neuronales recurrentes tradicionales, las LSTM pueden retener información durante largos periodos de tiempo, lo que las hace útiles para tareas como el procesamiento del lenguaje natural y la predicción de series temporales.

\subsection{Modelos de Lenguaje Grande (LLMs):} Los LLMs son modelos de lenguaje a gran escala que han sido entrenados en vastas cantidades de texto para comprender y generar lenguaje humano de manera coherente. Utilizan arquitecturas avanzadas de redes neuronales para aprender patrones complejos y relaciones en el lenguaje, permitiendo aplicaciones como la traducción automática, generación de texto y asistencia en tareas de escritura.


\subsection{Machine Learning (ML)}: Se trata de una rama de la Inteligencia Artificial que capacita a los sistemas para aprender y mejorar a partir de la experiencia, sin necesidad de ser programados explícitamente para cada tarea. Los algoritmos de ML son fundamentales para el funcionamiento de los chatbots, ya que permiten mejorar la precisión y relevancia de las respuestas basadas en datos históricos.

\subsection{Redes Neuronales y Deep Learning}: Las redes neuronales, especialmente las redes neuronales profundas (deep learning), han revolucionado el campo del NLP. Modelos como Transformers y BERT (Bidirectional Encoder Representations from Transformers) son capaces de comprender el contexto y matices del lenguaje humano con gran precisión.

\subsection{Telemedicina}: La telemedicina se refiere al uso de tecnologías de la información para proporcionar servicios médicos a distancia. Los chatbots médicos son una extensión de la telemedicina, proporcionando asesoramiento y orientación médica sin la necesidad de una interacción cara a cara.


\subsection{Salud Digital}: La salud digital engloba todas las aplicaciones de las tecnologías digitales en el ámbito sanitario. Esto incluye desde aplicaciones móviles hasta sistemas complejos de gestión de datos de pacientes. Los chatbots médicos se sitúan en la intersección de la salud digital y la telemedicina, facilitando el acceso a información médica de calidad.

\subsection{Interfaz de Usuario: } Intuitiva y amigable para padres y cuidadores, con consideraciones especiales para usuarios con bajo nivel de alfabetización tecnológica.


\subsection{Mecanismos de Seguridad} Encriptación de datos, autenticación de usuarios, y políticas de privacidad claras.